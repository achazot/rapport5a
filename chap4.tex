\chapter{Bilan et retour d'expérience}

	\section{Qualité des résultats}
	
		\content[inline]{A faire}

	\section{Perspectives d'évolution}
	
		\content[inline]{A faire}

	\section{Impact du projet}
		
		->nexter robotics
		\content[inline]{A faire}
	
	\section{Bilan personnel}
	
		\content[inline]{A faire}

	% 
	% \section{Pérennisation du projet}
	% 
	% 	\subsection{Manuel utilisateur et documentation}
	% 	\label{sub:manuel}
	% 
	% 		Le projet est destiné à être poursuivi lors de stages ultérieurs, c'est pourquoi il a été décidé de rédiger un manuel d'utilisation, autant à l'intention des développeurs que pour un utilisateur \og opérateur \fg{}. Il regroupe un manuel d'installation, un guide de résolution des problèmes, un manuel d'utilisation de l'interface homme-machine et un guide de développement, qui explique le fonctionnement du logiciel et propose des améliorations possibles.
	% 		\par
	% 		Ce manuel a été rédigé en Markdown \cite{markdown}, un langage de balisage léger permettant une mise en forme rapide du texte et l'inclusion d'images. Il est compilé sous la forme d'un site html statique par MkDocs\cite{mkdocs}, ce qui permet une mise en forme agréable, ainsi qu'une navigation simple et plus intuitive qu'un document \og classique \fg{}. Il est donc prêt à être hébergé sur un serveur web.
	% 		\par
	% 		Une documentation exhaustive de la partie \gls{ihm} a été écrite.
	% 		
	% 	\subsection{Installation automatique}
	% 	
	% 		\content[inline]{A faire}
	% 			
	% 	\subsection{Communication interne}
	% 	
	% 		\content[inline]{A faire}