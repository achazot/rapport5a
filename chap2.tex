\chapter{Un système de détection omnidirectionnel en temps réel}

	\section{Analyse du besoin}

		\subsection{Le concept du système}
			
			% Idée globale du système
			Le projet vise à crée un démonstrateur d'un concept plus large, à savoir un système permettant de piloter ensemble de robots et d'exploiter les informations issus de leurs périphériques de capture afin d'effectuer des missions de reconnaissance sans déployer de troupes au sol, et risquer de mettre leurs vies en dangers sur des terrains susceptibles d'être dangereux. Ce démonstrateur illustre donc la faisabilité de ce concept, mais d'une manière plus limitée. Le but est de développer un système opérationnel pour un seul robot, équipé d'un \gls{lidar} et d'une caméra sphérique.
			\par
			% Mon sujet de stage

		\subsection{Exigences}

			\lipsum[21-22]

	\section{Etat de l'art}

		\subsection{Réseaux neuronaux et apprentissage profond}
			
			\lipsum[23-24]

		\subsection{RPN et CNN}

			\lipsum[23-24]

		\subsection{Vidéo sphérique}
			% de 1 à 36 objectifs
			La photographie sphérique, aussi connue sous le nom de photographie à 360\degre ou \emph{VR photography} (pour réalité virtuelle), s'apparente à la photographie panoramique, en ceci qu'elle vise à capturer un point de vue sous la forme d'une image avec un champ de vision exceptionnellement large (ratio supérieur à $1:2$)\cite{fnumpano}. En effet, le but est de représenter une scène complète dans une seule image, comme on pourrait l'observer en effectuant un tour complet autour d'un point fixe. Le concept s'est fortement démocratisé au travers de son apparition dans \emph{Google StreetView}, et plus récemment par l'importante médiatisation de la réalité virtuelle, permettant une plus grande immersion lors du visionnage de photographie ou de vidéos sphériques.
			\par
			L'obtention d'une image prête au visionnage n'est théoriquement pas possible avec une seule prise de vue. Les techniques utilisées pour obtenir de telles images reposent toutes sur l'assemblage de plusieurs photographies, ce qui peut faire apparaître des incohérences dans la scène lorsqu'elles sont prises à des temps différents (les sujets qui se déplacent peuvent être présents sur plusieurs photographies, donc plusieurs parties de la scène). Pour obtenir le plus de cohérence dans le flux vidéo et minimiser le traitement dû à l'assemblage, il a donc été décidé d'acquérir un appareil qui permet de capturer instantanément une scène complète avec plusieurs objectifs.
			\par
			Cette médiatisation importante de la réalité virtuelle a entraîné la conception de caméras à 360\degre par de nombreux fabricants grand-public (LG, Samsung, Kodak), par opposition aux fabricants de matériel vidéo professionnel, comme FLIR (avec sa série des \emph{Ladybug} \cite{ladybug}). La recherche de matériel à acquérir s'est donc concrétisée par la création d'un comparatif des différentes caméras à 360\degre présent dans l'annexe X.
			
			Le choix d'acquisition du matériel à été piloté principalement par les facteurs suivants:
			\begin{itemize}[noitemsep,label=\textbullet]
				\item Couverture spatiale complète ($360^{\circ}\times180^{\circ}$)
				\item Au moins 15 images/secondes
				\item Transmission de la vidéo par WiFi ou USB
				\item Transmission de la capture vidéo en temps réel
				\item Assemblage d'image en temps réel
				\item Qualité d'image correcte (subjectif)
				\item Point de montage par vis
				\item Prix ne dépassant pas 500\euro
			\end{itemize}
			
			Notre choix s'était d'abord porté sur le produit Insta360 4K\cite{insta360}, qui possède un système d'assemblage en temps réel intégré, mais le produit n'étant pas disponible au moment de l'achat, nous avons dûs nous reporter sur la caméra Theta S de Ricoh\cite{ricohthetas}, possédant une qualité d'image moindre et aucun moyen natif d'assemblage en temps réel sous systèmes linux.

	\section{Problématiques soulevées}

		\subsection{Détection sur une image déformée}

			\lipsum[27-29]

		\subsection{Visualisation des résultats}

			\lipsum[30-32]
		
			
