\section*{Introduction}
\phantomsection
\addcontentsline{toc}{section}{Introduction}
	
	% Contexte
	Ce document relate le travail qui a été effectué au sein du pôle recherche et développement de l'agence de SII Bourges, dans le cadre d'un stage de fin d'études à l'issue de la formation en Sécurité et Technologies Informatique de l'INSA Centre Val de Loire, option Architecture et Sécurité Logicielle.
	
	% Place du projet dans l'entreprise
	La mission s'inscrit dans la continuité du projet robotique du pôle R\&D en coopération avec l'INSA Centre Val de Loire, dont un précédant stage avait fait l'objet. Cependant, il convient de préciser que le produit présenté ne se base pas sur le précédent travail, si ce n'est dans l'esprit de sa réalisation et de ses objectifs à long terme. En effet, il a pour vocation de faire croître le savoir-faire du pôle face aux problématiques appliquées du traitement du signal et de mettre en avant ces compétences devant des acteurs clés.
	
	% Description du projet
	Le projet est une preuve de concept, dont la finalité répond à la dénomination \emph{Système Robotique Tactique Multi-Missions pour la Surveillance et l'Aide à la Prise de Décision en Milieux à Risques} (\gls{robo}). Le système opérationnel final permet de recueillir des informations stratégiques sur l'environnement où il sera déployé, sans mettre en danger des acteurs humains. Ce \gls{poc} a fait l'objet de deux stages dont les sujets sont étroitements liés. Le premier, qui vise à effectuer une cartographie des lieux grâce à un \gls{lidar}, ne sera pas détaillé dans ce rapport. La mission décrite ici consiste principalement à concevoir et implémenter un système de détection et classification d'objets d'intérêt dans un flux vidéo sphérique, ainsi qu'une brique logicielle servant d'interface à la visualisation de ce flux vidéo agrémenté des résultats de la détection.
	
	% Présentation du plan
	Nous consacrerons une première partie à préciser le contexte de réalisation de ce stage, de la création de l'entreprise SII au partenariat avec l'INSA CVL. La deuxième partie proposera un point sur l'expression du besoin et sur les technologies nécessaires à sa réalisation, pour ensuite décrire la manière dont à été conçu le système dans une troisième partie. Pour finir, la quatrième partie fera état des résultats obtenus et concluera sur l'intérêt du stage, autant du point de vue personnel que global.
