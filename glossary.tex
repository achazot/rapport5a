% Magic here
\makeglossaries

\newglossaryentry{usb}
{
	name={USB},
	description={\emph{Universal Serial Bus}, norme de bus informatique en transmission série}
}

\newglossaryentry{uvc}
{
	name={UVC},
	description={\emph{\gls{usb} Video Class}, protocole de gestion de périphériques électroniques vidéo via les ports de type \gls{usb}}
}

\newglossaryentry{mjpeg}
{
	name={MJPEG},
	description={\emph{Motion \gls{jpeg}}, codec vidéo qui compresse les images une à une en \gls{jpeg}.}
}

\newglossaryentry{jpeg}
{
	name={JPEG},
	description={\emph{Joint Photographic Experts Group}, norme de compression d'images à perte basée sur le sous-échantillonage de la chrominance, la transformée en cosinus discrète, et une compression RLE/Huffman}
}

\newglossaryentry{ros}
{
	name={ROS},
	description={\emph{Robot Operating System}, framework dédié à la robotique, possédant entre autre une importante couche d'abstraction des communications entre programmes}
}

\newglossaryentry{cuda}
{
	name={CUDA},
	description={\emph{Compute Unified Device Architecture}, bibliothèque développée par NVIDIA permettant de programmer des \gls{gpu} en langage C}
}


\newglossaryentry{lidar}
{
	name={LIDAR},
	description={\emph{LIght Detection And Ranging}, une technique de télémétrie (détermination de la distance d'un objet) qui se base sur la mesure du temps écoulé entre l'émission et la réception d'un faisceau de lumière. Dans notre cas, la source de lumière est un laser émettant de l'infrarouge. Par extension, on nomme LIDAR le dispositif matériel permettant de telles mesures}
}

\newglossaryentry{slam}
{
	name={SLAM},
	description={\emph{Simultaneous Localization And Mapping}, technique qui consiste, pour un robot ou véhicule autonome, à simultanément construire ou améliorer une carte de son environnement et de s’y localiser}
}

\newglossaryentry{gpu}
{
	name={GPU},
	description={\emph{Graphic Processing Unit}, Processeur possédant une structure hautement parallèle, permettant des calculs matriciels beaucoup plus rapides qu'avec un processeur possédant une architecture plus classique (CPU) et de ce fait utilisé principalement pour les traitements graphiques tels que le rendu 3D}
}

\newglossaryentry{poc}
{
	name={POC},
	description={\content{à développer}\emph{Proof Of Concept}}
}

\newglossaryentry{bu}
{
	name={BU},
	description={\emph{Business Unit}, unité organisationnelle au sein du groupe SII s'adressant à un secteur d'activité particulier}
}

\newglossaryentry{b2b}
{
	name={B2B},
	description={\emph{Business to Business}, caractérise les activités d'une entreprise visant une clientèle d'entreprises}
}

\newglossaryentry{rd}
{
	name={R\&D},
	description={\emph{Recherche et Developpement}, caractérise les activités d'une entreprise visant à accroître ses connaissances au travers de la recherche fondamentale et appliquée et à effectuer des réalisations expérimentales de manière à créer des produits ou stratégies innovants}
}

\newglossaryentry{robo}
{
	name={SRT2M},
	description={\emph{Système Robotique Tactique Multi-Mission}, nom du produit final porté par le projet dont il est question dans ce rapport, et par extension le démonstrateur lui-même}
}
